%!TEX TS-program = xelatex
%!TEX encoding = UTF-8 Unicode
% Awesome CV LaTeX Template for CV/Resume
%
% This template has been downloaded from:
% https://github.com/posquit0/Awesome-CV
%
% Author:
% Claud D. Park <posquit0.bj@gmail.com>
% http://www.posquit0.com
%
% Template license:
% CC BY-SA 4.0 (https://creativecommons.org/licenses/by-sa/4.0/)
%


%-------------------------------------------------------------------------------
% CONFIGURATIONS
%-------------------------------------------------------------------------------
% A4 paper size by default, use 'letterpaper' for US letter
\documentclass[11pt, a4paper]{awesome-cv}


\newcommand*\sq{\mathbin{\vcenter{\hbox{\rule{.3ex}{.3ex}}}}}
\newcommand*\sqbig{\hspace{0.5em}\mathbin{\vcenter{\hbox{\rule{.3ex}{.3ex}}}}\hspace{0.5em}}

% Configure page margins with geometry
\geometry{left=1.4cm, top=.8cm, right=1.4cm, bottom=1.8cm, footskip=.5cm}

% Specify the location of the included fonts
\fontdir[fonts/]

% Color for highlights
% Awesome Colors: awesome-emerald, awesome-skyblue, awesome-red, awesome-pink, awesome-orange
%                 awesome-nephritis, awesome-concrete, awesome-darknight
\colorlet{awesome}{awesome-red}
% Uncomment if you would like to specify your own color
% \definecolor{awesome}{HTML}{CA63A8}

% Colors for text
% Uncomment if you would like to specify your own color
% \definecolor{darktext}{HTML}{414141}
% \definecolor{text}{HTML}{333333}
% \definecolor{graytext}{HTML}{5D5D5D}
% \definecolor{lighttext}{HTML}{999999}
% \definecolor{sectiondivider}{HTML}{5D5D5D}

% Set false if you don't want to highlight section with awesome color
\setbool{acvSectionColorHighlight}{true}

% If you would like to change the social information separator from a pipe (|) to something else
\renewcommand{\acvHeaderSocialSep}{\quad\textbar\quad}


%-------------------------------------------------------------------------------
%	PERSONAL INFORMATION
%	Comment any of the lines below if they are not required
%-------------------------------------------------------------------------------
% Available options: circle|rectangle,edge/noedge,left/right
%\photo[left]{../../../images/grum.png}
\name{Romain}{Gallet}
\position{software architecture $\sq$ data engineering $\sq$ devops $\sq$ aws certified}
%\address{42-8, Bangbae-ro 15-gil, Seocho-gu, Seoul, 00681, Rep. of KOREA}

%\cpyname{Grum}{Ltd}

\mobile{(+352) 661\,54\,72\,73}
\email{romain.gallet@gmail.com}
%\homepage{grumlimited.co.uk}
\github{grumlimited}
\linkedin{romaingallet}
% \gitlab{gitlab-id}
% \stackoverflow{SO-id}{SO-name}
% \twitter{@twit}
% \skype{skype-id}
% \reddit{reddit-id}
% \medium{medium-id}
% \kaggle{kaggle-id}
% \googlescholar{googlescholar-id}{name-to-display}
%% \firstname and \lastname will be used
% \googlescholar{googlescholar-id}{}
% \extrainfo{extra information}

%\quote{ %\textcolor{awesome}{\textbf{CONTRACTS ONLY - LONDON ONLY}}\\
%\vspace{1em}
% Engineering and consultancy services within the media, financial and publishing industries.\\
%\textbf{Grum Ltd} provides expertise in high availability architectures around streaming data pipelines.
%}

\hypersetup{%
  pdftitle={Romain GALLET~~~·~~~Consultant},
  pdfauthor={Romain Gallet},
  pdfsubject={Romain GALLET~~~·~~~Consultant},
  pdfkeywords={CV, Consultant}
}

%-------------------------------------------------------------------------------
\begin{document}

% Print the header with above personal information
% Give optional argument to change alignment(C: center, L: left, R: right)
\makecvheader

% Print the footer with 3 arguments(<left>, <center>, <right>)
% Leave any of these blank if they are not needed
\makecvfooter
  %{\today}
  {}
  {Romain GALLET~~~·~~~Consultant}
  {\thepage}


%-------------------------------------------------------------------------------
%	CV/RESUME CONTENT
%	Each section is imported separately, open each file in turn to modify content
%-------------------------------------------------------------------------------
%-------------------------------------------------------------------------------
%	SECTION TITLE
%-------------------------------------------------------------------------------
\cvsection{Services}


%-------------------------------------------------------------------------------
%	CONTENT
%-------------------------------------------------------------------------------
\begin{cvparagraph}

%---------------------------------------------------------
15+ years of expertise in designing and implementing high-availability apis and data architectures, enhancing client infrastructures to meet strategic fast and big data objectives.
Specialised in devops methodologies, utilising Kubernetes, Terraform, Kafka, and Kinesis to productionise cloud-based ETLs and real-time streaming processes.
Proficient in Java, Scala and Rust software development, crafting robust applications with Akka, FS2, Kafka Streams, and Spark, from initial requirements to final deployment.
Engineering CI/CD pipelines to ensure the rapid, well tested, reliable delivery of complex software systems, reinforcing a culture of continuous improvement and innovation.
\end{cvparagraph}

%-------------------------------------------------------------------------------
%	SECTION TITLE
%-------------------------------------------------------------------------------
\cvsection{Specialisations}

%-------------------------------------------------------------------------------
%	CONTENT
%-------------------------------------------------------------------------------
\begin{cvskills}

%---------------------------------------------------------
    \cvskill
    {Programming} % Category
    {java {} scala {} groovy {} rust {} go {} python {} cats {} fs2 {} http4s {} play {} spring}

%---------------------------------------------------------
    \cvskill
    {Backend} % Category
    {kafka {} kinesis {} ksqldb {} mongodb {} dynamodb {} rdbms {} amqp {} jms}

%---------------------------------------------------------
    \cvskill
    {Observability} % Category
    {contract testing {} opentelemetry {} datadog {} smithy {} openapi {} swagger}

%---------------------------------------------------------
    \cvskill
    {Devops} % Category
    {aws {} docker {} kubernetes {} terraform {} cloudformation {} jenkins {} circleci {} argocd}

%---------------------------------------------------------
    \cvskill
    {Data engineering} % Category
    {etl {} spark {} spark\,streaming {} kafka\,streams {} akka\,streams {} uml {} flink {} pii {} gdpr}

%---------------------------------------------------------

    \cvskill
    {Networks} % Category
    {http {} dns {} ntp {} qemu {} ssl {} routing {} firewall }

%---------------------------------------------------------
\end{cvskills}

%-------------------------------------------------------------------------------
%	SECTION TITLE
%-------------------------------------------------------------------------------
\cvsection{Consulting}

\newcommand*{\logo}[2]{\raisebox{-0.2em}{\includegraphics[height=1em]{#2}}\hspace{0.25em}#1}
\newcommand*{\logoonly}[1]{\raisebox{-0.2em}{\includegraphics[height=1em]{#1}}}


%-------------------------------------------------------------------------------
%	CONTENT
%-------------------------------------------------------------------------------
\begin{cventries}
  \cventry
    {Scala\,|\,java\,|\,data engineer}
    {\logo{Disney}{../../../images/disney.png}} % Organization
    {Luxembourg} % Location
    {October 2022 - present} % Date(s)
    {
      Architectured a solution able to manage over 700 million users, ensuring seamless auto-scaling to accommodate 150,000 requests per second. Engineered Single Sign-On (SSO) authentication backend layers across multiple systems, leveraging Fs2, and utilising Kinesis and DynamoDB streams for state management and materialised views creation. 
      \newline
      Developed a Pairwise Pseudonymous Identifiers (PPID) system utilising http4s and DynamoDB, enhancing user privacy through unique, opaque identifiers for third-party client-specific interactions.
    }

%---------------------------------------------------------
  \cventry
    {Scala\,|\,java\,|\,data engineer}
    {\logo{Depop}{../../../images/depop.jpg}} % Organization
    {Luxembourg} % Location
    {March 2022 - September 2022} % Date(s)
    {
       Engineered KSQL and Spark Streaming data pipelines, delivering enriched streaming and batch datasets with optimised sub 15-minute materialised views (Delta tables) for efficient OLAP queries utilising dbt, Spark, and Athena.
    }

%---------------------------------------------------------

  \cventry
    {Scala\,|\,java\,|\,data engineer}
    {\logo{Zego}{../../../images/zego.jpg}} % Organization
    {Luxembourg} % Location
    {Nov 2021 - Aug 2022} % Date(s)
    {
     Engineered architecture integrating third-party vehicle and insurance data via REST APIs, gRPC, FTP, and WebSockets, leveraging Kinesis and Akka-Streams for ingress and egress data flows. Orchestrated CI/CD pipelines utilising Kubernetes, ArgoCD, and Buildkite to ensure deployment and scaling of microservices and stream processors.
    }

%---------------------------------------------------------

  \cventry
    {Scala\,|\,java\,|\,data engineer}
    {\logo{Slice}{../../../images/slice.png}} % Organization
    {Luxembourg} % Location
    {Apr 2021 - Oct 2021} % Date(s)
    {
    Spearheaded migration to a high-throughput, stream-oriented data processing architecture, enhancing system scalability and performance. Provided expert guidance and led the development of robust stream processors on AWS and Confluent platforms. Engineered CI/CD pipelines for streaming data (Kafka Streams, Spark Streaming, ksqlDB) from Databricks to Confluent, ensuring seamless deployment and integration.
    }

%---------------------------------------------------------
  \cventry
    {Scala\,|\,java\,|\,data engineer}
    {\logo{Compare The Market}{../../../images/ctm.png}} % Organization
    {London} % Location
    {Jan 2020 - Mar 2021} % Date(s)
    {
    Designed the architecture and technical strategy for advanced streaming infrastructure, effectively engaging with Architecture Design Forums (ADF) and synchronising with key stakeholders. Defined and implemented timelines and workflows for the deployment of proposed technical solutions. Orchestrated CI/CD pipelines for streaming data flows utilising Kafka Streams, Akka Streams, and Spark Streaming on EMR, EC2, EKS, and MSK.
    }

\end{cventries}

\cvsection{Other clients}

%\begin{table}[h]
%	\centering
%	\begin{tabular*}{\columnwidth}{@{\extracolsep{\fill}}lllll}
%		{\entrytitlestyle{\logo{Depop}{../../../images/depop.jpg}}} & {\entrytitlestyle{\logo{Guardian}{../../../images/guardian.png}}} & {\entrytitlestyle{{\logoonly{../../../images/voa.png}}}} & {\entrytitlestyle{\logo{Sainsbury's}{../../../images/sainsburys.jpg}}} & {\entrytitlestyle{{\logo{Capco}{../../../images/capco.png}}}} \\[8pt]
%		{\entrytitlestyle{{\logoonly{../../../images/newsuk.png}}}} & {\entrytitlestyle{HappyInc}} & {\entrytitlestyle{ {\logoonly{../../../images/sky.jpeg}}}} & {\entrytitlestyle{{\logoonly{../../../images/ho.png}}}} & {\entrytitlestyle{{\logo{DMC Digital}{../../../images/dealchecker.jpg}}}} \\[8pt]
%		{\entrytitlestyle{{\logo{ITHR}{../../../images/ithr.jpg}}}} & {\entrytitlestyle{{\logo{TripAdvisor}{../../../images/tripadvisor.jpg}}}} & {\entrytitlestyle{{\logo{SixAndCo}{../../../images/sixandco.png}}}} & {\entrytitlestyle{{\logo{VoxMobili}{../../../images/voxmobili.png}}}} & {\entrytitlestyle{{\logo{Fullsix}{../../../images/fullsix.jpg}}}}
%	\end{tabular*}
%\end{table}

 \begin{flushleft}
{\entrytitlestyle{\logo{Depop}{../../../images/depop.jpg}}}
$\sqbig$ 
{\entrytitlestyle{\logo{Guardian}{../../../images/guardian.png}}}
$\sqbig$
{\entrytitlestyle{{\logoonly{../../../images/voa.png}}}}
$\sqbig$ 
{\entrytitlestyle{\logo{Sainsbury's}{../../../images/sainsburys.jpg}}}
$\sqbig$ 
{\entrytitlestyle{{\logo{Capco}{../../../images/capco.png}}}}
$\sqbig$ 
{\entrytitlestyle{{\logoonly{../../../images/newsuk.png}}}}
$\sqbig$ 
{\entrytitlestyle{HappyInc}}
$\sqbig$
{\entrytitlestyle{ {\logoonly{../../../images/sky.jpeg}}}}
$\sqbig$
{\entrytitlestyle{{\logoonly{../../../images/ho.png}}}}  
$\sqbig$

{\entrytitlestyle{{\logo{DMC Digital}{../../../images/dealchecker.jpg}}}}
$\sqbig$
{\entrytitlestyle{{\logo{ITHR}{../../../images/ithr.jpg}}}}
$\sqbig$ 
{\entrytitlestyle{{\logo{TripAdvisor}{../../../images/tripadvisor.jpg}}}}
$\sqbig$ 
{\entrytitlestyle{{\logo{SixAndCo}{../../../images/sixandco.png}}}}
$\sqbig$ 
{\entrytitlestyle{{\logo{VoxMobili}{../../../images/voxmobili.png}}}}
$\sqbig$ 
{\entrytitlestyle{{\logo{Fullsix}{../../../images/fullsix.jpg}}}}

\end{flushleft}
%\cvsection{Consulting}


%---------------------------------------------------------
%  \cventry
%    {Scala and data engineer} 
%    {\logo{Depop}{../../../images/depop.jpg}} % Organization
%    {London} % Location
%    {June 2019 - Dec 2019} % Date(s)
%    {}
%      \begin{cvitems}
%        \item{Designed streamed aws redshift and rabbitmq data ingestion pipelines with kinesis and akka-streams data processors. }
%	\item{Coordinating and coaching for teams members around streaming data processes and technologies.}
%	\item{Research and identification of opportunities (kafka streams, samza), developing proposals and recommendations.}
%      \end{cvitems}     

%---------------------------------------------------------
%  \cventry
%    {Scala and data engineer} 
%    {\logo{The Guardian}{../../../images/guardian.png}} % Organization
%    {London} % Location
%    {Oct 2018 - May 2019} % Date(s)
%    {}
	%\begin{cvitems}
	%	\item{Implementation of spark etl pipelines on aws emr, processing readers' revenue and subscriptions datasets.}
	%	\item{Designed and built alternative streaming pipelines with aws msk (kafka), eks and kafka-streams. Training and coaching around streaming data pipelines and technologies.}
	%	\item{Implemented in-memory data stream uploads using http4s and fs2 to converse with rest apis.}
	%\end{cvitems}     

%---------------------------------------------------------
%  \cventry
%    {Scala and data engineer} 
%    {\logo{Sainsbury's}{../../../images/sainsburys.jpg}} % Organization
%    {London} % Location
%    {Dec 2017 - Sep 2018} % Date(s)
%    {}
      %\begin{cvitems}
      %  \item{Replaced nectar cards batch-based data processing with streaming pipelines ingesting and processing data in 4 seconds or less.}
%	\item{Architecture and deployment of kafka clusters (apache and cloudera), kms and kerberos.}
	%\item{Design and implementation of kafka-streams, kafka-connect, and schema-registry etls on aws ecs. Data analytics using hive, spark and zeppelin.}
	%\item{Implementation of pii and gdpr-compliant stream processors using data tokenisation and encryption.}
  %    \end{cvitems}     

%---------------------------------------------------------
%  \cventry
%    {Scala engineer} 
%    {\logo{Home Office (HMPO \& IPT)}{../../../images/ho.png}} % Organization
%    {London} % Location
%    {June 2016 - Nov 2017} % Date(s)
%    {}
    %  \begin{cvitems}
    %    \item{Development of HMPO's online passport renewals portal. Designed data ingestion processes.}
%	\item{Implementation of resilient data transfer layers with akka-streams and akka-persistence on kubernetes.}
	%\item{Api contract testing using swagger and gatling.}
   %   \end{cvitems}     

%---------------------------------------------------------
%  \cventry
%    {Scala engineer} 
%    {\logo{HMRC (Valuation Office Agency)}{../../../images/voa.png}} % Organization
%    {London} % Location
%    {Nov 2015 - May 2016} % Date(s)
%    {}
    %  \begin{cvitems}
    %    \item{Design and implementation of an akka wrapper around a legacy web application with akka-streams. Proposed design to stakeholders.}
	%\item{Delivered a gds-compliant solution that turned a seldom-used service to a country-wide first-choice platform for council tax and business rates collections.}
  %    \end{cvitems}     


%---------------------------------------------------------
%  \cventry
%    {Scala engineer} 
%    {\logo{Home Office (HMPO \& Border Force)}{../../../images/ho.png}} % Organization
%    {London} % Location
%    {Mars 2014 - Oct 2015} % Date(s)
%    {}
%      \begin{cvitems}
%        \item{Design and implementation of a flight arrivals prediction engine. Developed resiliency and recovery procedures using akka and rabbitmq.}
%	\item{Design and implementation of internal tools for case worker application management, written using play.}
%      \end{cvitems}     
    

%---------------------------------------------------------
%  \cventry
%    {Full stack jee engineer} 
%    {\logo{Capco}{../../../images/capco.png}} % Organization
%    {London} % Location
%    {Sep 2013 - Dec 2013} % Date(s)
%    {}

%---------------------------------------------------------
%  \cventry
%    {Full stack jee/grails engineer} 
%    {\logo{News UK (News International)}{../../../images/newsuk.png}} % Organization
%    {London} % Location
%    {Nov 2012 - June 2013} % Date(s)
%     {}

%---------------------------------------------------------
%  \cventry
%    {Full stack jee/grails engineer} 
%    {\logo{HappyInc Ltd {\textit (now Blow Ltd)}}{../../../images/blowltd.jpg}} % Organization
%    {London} % Location
%    {July 2012 - Sep 2012} % Date(s)
%    {}

%---------------------------------------------------------
%  \cventry
%    {Full stack jee/grails engineer} 
%    {\logo{BSkyB (Sky Group)}{../../../images/sky.jpeg}} % Organization
%    {London} % Location
%    {Aug 2011 - June 2012} % Date(s)
%    {}

%---------------------------------------------------------
%  \cventry
%    {Senior jee engineer} 
%    {\logo{DMC Digital}{../../../images/dealchecker.jpg}} % Organization
%    {London} % Location
%    {Dec 2010 - July 2011} % Date(s)
%    {}

%---------------------------------------------------------
%  \cventry
%    {Senior jee engineer} 
%    {\logo{ITHR}{../../../images/ithr.jpg}} % Organization
%    {London} % Location
%    {Oct 2010 - Nov 2010} % Date(s)
%    {}
%---------------------------------------------------------

%-------------------------------------------------------------------------------
%	SECTION TITLE
%-------------------------------------------------------------------------------
%\cvsection{Permanent positions}

%-------------------------------------------------------------------------------
%	CONTENT
%-------------------------------------------------------------------------------
%\begin{cventries}

%---------------------------------------------------------
%  \cventry
%    {Technical lead - full stack jee engineer} 
%    {\logo{TripAdvisor}{../../../images/tripadvisor.jpg}} % Organization
%    {London} % Location
%    {June 2008 - Aug 2010} % Date(s)
%    {}

%---------------------------------------------------------
%  \cventry
%    {Technical lead - full stack jee engineer} 
%    {\logo{SixAndCo (Fullsix group)}{../../../images/sixandco.png}} % Organization
%    {London} % Location
%    {Dec 2006 - Apr 2008} % Date(s)
%    {}

%---------------------------------------------------------
%  \cventry
%    {Java engineer} 
%    {\logo{VoxMobili (Anéo)}{../../../images/voxmobili.png}} % Organization
%    {Paris} % Location
%    {Sep 2006 - Nov 2006} % Date(s)
%    {}

%---------------------------------------------------------
%  \cventry
%    {Full stack java engineer} 
%    {\logo{Fullsix}{../../../images/fullsix.jpg}} % Organization
%    {London} % Location
%    {Aug 2004 - June 2005} % Date(s)
%    {}

%---------------------------------------------------------
%  \cventry
%    {Junior full stack java engineer} 
%    {BoxNewMedia} % Organization
%    {London} % Location
%    {July 2003 - July 2004} % Date(s)
%    {}

%---------------------------------------------------------
%  \cventry
%    {Junior full stack java engineer} 
%    {AVS Consulting} % Organization
%    {London} % Location
%    {Mar 2002 - July 2002} % Date(s)
%    {}

% \end{cventries}

\newpage

%\newpage
%-------------------------------------------------------------------------------
%	SECTION TITLE
%-------------------------------------------------------------------------------
\cvsection{Education}


%-------------------------------------------------------------------------------
%	CONTENT
%-------------------------------------------------------------------------------
\begin{cventries}

%---------------------------------------------------------
  \cventry
    {master in business management} % Degree
    {\raisebox{-0.3em}{\includegraphics[height=1em]{../../../images/hec.png}}\,\raisebox{-0.2em}{\includegraphics[height=1em]{../../../images/telecomparis.jpg}} HEC \& Télécom Paris (ENST)} % Institution
    {Paris} % Location
    {2006} % Date(s)
    {}
    {}

%---------------------------------------------------------
  \cventry
    {msc in computer sciences, 1\textsuperscript{st} class with distinction} % Degree
    {\raisebox{-0.1em}{\includegraphics[height=1em]{../../../images/greenwich.jpg}}\hspace{0.5em}Greenwich University} % Institution
    {London} % Location
    {2004} % Date(s)
    {}
    {}

%---------------------------------------------------------
  \cventry
    {master isri (ingénierie des systèmes et des réseaux informatiques), 1\textsuperscript{st} class} % Degree
    {\raisebox{-0.2em}{\includegraphics[height=1em]{../../../images/upjv.jpg}}\hspace{0.5em}Université de Picardie Jules Verne} % Institution
    {Paris} % Location
    {2003} % Date(s)
    {}
    {}

%---------------------------------------------------------
  \cventry
    {bsc in computer sciences} % Degree
    {\raisebox{-0.2em}{\includegraphics[height=1em]{../../../images/bham.png}}\hspace{0.5em}Birmingham University} % Institution
    {London} % Location
    {2002} % Date(s)
    {}
    {}
%---------------------------------------------------------
\end{cventries}

%-------------------------------------------------------------------------------
%	SECTION TITLE
%-------------------------------------------------------------------------------
\cvsection{Certifications}


%-------------------------------------------------------------------------------
%	SUBSECTION TITLE
%-------------------------------------------------------------------------------
%\cvsubsection{International}


%-------------------------------------------------------------------------------
%	CONTENT
%-------------------------------------------------------------------------------


\begin{cventries}

%---------------------------------------------------------
    \cventry
    {\href{https://www.credly.com/badges/e5761b98-edf5-44ec-bb32-24f18672e7de/public_url}{aws certified solutions architect – associate}}
    {\raisebox{-0.2em}{\includegraphics[height=1em]{../../../images/aws.jpg}}\hspace{0.5em}Amazon AWS} % Event
    {}
    {2021}
    {}
    {}
    {}

    \cventry
    {\href{https://www.coursera.org/account/accomplishments/specialization/9NNUCHVNV36F}{functional programming in scala}}
    {\raisebox{-0.2em}{\includegraphics[height=1em]{../../../images/epfl.png}}\hspace{0.5em}École Polytechnique Fédérale de Lausanne} % Event
    {}
    {2017}
    {}
    {}
    {}

    \cventry
    {oracle certification for java programmer (ocjp)} % Award
    {\raisebox{-0.2em}{\includegraphics[height=1em]{../../../images/oracle.png}}\hspace{0.5em}Oracle} % Event
    {}
    {2012} % Date(s)
    {}
    {}
    {}


\end{cventries}

%%-------------------------------------------------------------------------------
%	SECTION TITLE
%-------------------------------------------------------------------------------
\cvsection{Presentation}


%-------------------------------------------------------------------------------
%	CONTENT
%-------------------------------------------------------------------------------
\begin{cventries}

%---------------------------------------------------------
  \cventry
    {Presenter for <Hosting Web Application for Free utilizing GitHub, Netlify and CloudFlare>} % Role
    {DevFest Seoul by Google Developer Group Korea} % Event
    {Seoul, S.Korea} % Location
    {Nov. 2017} % Date(s)
    {keywords, highlights, etc} % Highlights
    {
      \begin{cvitems} % Description(s)
        \item {Introduced the history of web technology and the JAM stack which is for the modern web application development.}
        \item {Introduced how to freely host the web application with high performance utilizing global CDN services.}
      \end{cvitems}
    }

%---------------------------------------------------------
  \cventry
    {Presenter for <DEFCON 20th : The way to go to Las Vegas>} % Role
    {6th CodeEngn (Reverse Engineering Conference)} % Event
    {Seoul, S.Korea} % Location
    {Jul. 2012} % Date(s)
    {} % Highlights
    {
      \begin{cvitems} % Description(s)
        \item {Introduced CTF(Capture the Flag) hacking competition and advanced techniques and strategy for CTF}
      \end{cvitems}
    }

%---------------------------------------------------------
\end{cventries}

%-------------------------------------------------------------------------------
%	SECTION TITLE
%-------------------------------------------------------------------------------
\cvsection{Other projects}


%-------------------------------------------------------------------------------
%	CONTENT
%-------------------------------------------------------------------------------
\begin{cventries}

%---------------------------------------------------------
  \cventry
    {\href{https://github.com/grumlimited/authenticator-rs}{github.com/grumlimited/authenticator-rs}} % Role
     {\raisebox{-0.15em}{\includegraphics[height=1em]{../../../images/github.png}}\hspace{0.2em}Authenticator-rs} % Title
    {} % Location
    {} % Date(s)
    {}
    {
MFA (multi-factors authentication) desktop (linux) application written in Rust and gtk3. Generates totp tokens to use with mfa-enabled security backends.
    }


  \cventry
    {\href{https://github.com/grumlimited/geocalc}{github.com/grumlimited/geocalc}} % Role
    {\raisebox{-0.15em}{\includegraphics[height=1em]{../../../images/github.png}}\hspace{0.2em}GeoCalc} % Title
    {} % Location
    {} % Date(s)
    {}
    {
Geocalc is a simple java library providing arithmetic tools to manipulate Earth coordinates. Allows for various distance, trajectory and surface calculations. Forms the basis for rentbarometer.com
to calculate geo-based properties prices.
    }

%  \cventry
%    {\href{https://www.rentbarometer.com}{rentbarometer.com}} % Role
%    {\raisebox{-0.2em}{\includegraphics[height=1em]{../../../images/rb.png}}\hspace{0.5em}Rentbarometer} % Title
%    {} % Location
%    {} % Date(s)
%    {}
%    {
%A data-driven website aiming to restore some fairness and balance in the property rental market in London.
%    }
\end{cventries}

%%-------------------------------------------------------------------------------
%	SECTION TITLE
%-------------------------------------------------------------------------------
\cvsection{Program Committees}


%-------------------------------------------------------------------------------
%	CONTENT
%-------------------------------------------------------------------------------
\begin{cvhonors}

%---------------------------------------------------------
  \cvhonor
    {Problem Writer} % Position
    {2016 CODEGATE Hacking Competition World Final} % Committee
    {S.Korea} % Location
    {2016} % Date(s)

%---------------------------------------------------------
  \cvhonor
    {Organizer \& Co-director} % Position
    {1st POSTECH Hackathon} % Committee
    {S.Korea} % Location
    {2013} % Date(s)

%---------------------------------------------------------
\end{cvhonors}

%%-------------------------------------------------------------------------------
%	SECTION TITLE
%-------------------------------------------------------------------------------
\cvsection{Extracurricular Activity}


%-------------------------------------------------------------------------------
%	CONTENT
%-------------------------------------------------------------------------------
\begin{cventries}

%---------------------------------------------------------
  \cventry
    {Core Member \& President at 2013} % Affiliation/role
    {PoApper (Developers' Network of POSTECH)} % Organization/group
    {Pohang, S.Korea} % Location
    {Jun. 2010 - Jun. 2017} % Date(s)
    {
      \begin{cvitems} % Description(s) of experience/contributions/knowledge
        \item {Reformed the society focusing on software engineering and building network on and off campus.}
        \item {Proposed various marketing and network activities to raise awareness.}
      \end{cvitems}
    }

%---------------------------------------------------------
  \cventry
    {Member} % Affiliation/role
    {PLUS (Laboratory for UNIX Security in POSTECH)} % Organization/group
    {Pohang, S.Korea} % Location
    {Sep. 2010 - Oct. 2011} % Date(s)
    {
      \begin{cvitems} % Description(s) of experience/contributions/knowledge
        \item {Gained expertise in hacking \& security areas, especially about internal of operating system based on UNIX and several exploit techniques.}
        \item {Participated on several hacking competition and won a good award.}
        \item {Conducted periodic security checks on overall IT system as a member of POSTECH CERT.}
        \item {Conducted penetration testing commissioned by national agency and corporation.}
      \end{cvitems}
    }

%---------------------------------------------------------
\end{cventries}



%-------------------------------------------------------------------------------
\end{document}
